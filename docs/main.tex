\documentclass[a4paper,11pt]{article}

\usepackage{cmap}		
%\usepackage[utf8]{inputenc}			
\usepackage[brazil]{babel}
\usepackage{framed}
\usepackage{hyperref}
\usepackage{amsmath}
\usepackage{graphicx}
\setlength {\marginparwidth }{2cm}
\usepackage[colorinlistoftodos]{todonotes}
\usepackage{wrapfig}
\usepackage{lipsum}
\usepackage{listings}
\usepackage{color}
\usepackage{indentfirst}
\usepackage{times}
\usepackage{textcomp}
\usepackage{pgfgantt}
\usepackage{lipsum}
\usepackage{xcolor}

% set document font, font sizes, margin dimensions and spacing
\usepackage{fontspec}
\setmainfont{Arial}
\usepackage[top=15mm,bottom=25mm,left=20mm,right=20mm]{geometry}
\usepackage{setspace}\onehalfspacing
\usepackage{titlesec}
\titleformat*{\section}{\Large\bfseries}
\titleformat*{\subsection}{\Large\bfseries}
\titleformat*{\subsubsection}{\Large\bfseries}
\titleformat*{\paragraph}{\Large\bfseries}
\titleformat*{\subparagraph}{\Large\bfseries}
\setlength{\parskip}{0.6em}

\newif\ifblackandwhite
\blackandwhitetrue

\usepackage{etoolbox}
\usepackage{longtable}%
\AtBeginEnvironment{longtable}{%
  \addfontfeature{RawFeature=+tnum;-onum}%  <--- requires LuaTeX
}

\usepackage{pdflscape}
%\usepackage[svgnames]{xcolor}
 \usepackage{colortbl}%
   \newcommand{\myrowcolour}{\rowcolor[gray]{0.925}}
\usepackage{booktabs}

\ifblackandwhite
  \newcommand{\cheading}[2]{\textbf{#1\hfill #2}}
  \newcommand{\highest}[1]{\textbf{#1}}% == highest score for question
\else
  \newcommand{\cheading}[2]{\textcolor{Maroon}{\textbf{#1\hfill #2}}}
  \newcommand{\highest}[1]{\textcolor{Maroon}{\textbf{#1}}}%
\fi

\definecolor{mygray}{rgb}{0.4,0.4,0.4}
\definecolor{mygreen}{rgb}{0,0.8,0.6}
\definecolor{myorange}{rgb}{1.0,0.4,0}

\lstdefinestyle{customc}{
  belowcaptionskip=1\baselineskip,
  breaklines=true,
  frame=L,
  xleftmargin=\parindent,
  language=C,
  showstringspaces=false,
  basicstyle=\footnotesize\ttfamily,
  keywordstyle=\bfseries\color{green!40!black},
  commentstyle=\itshape\color{purple!40!black},
  identifierstyle=\color{blue},
  stringstyle=\color{orange},
  numbers=left,
  numbersep=12pt,
  numberstyle=\small\color{mygray},
}
\lstset{escapechar=@,style=customc}

\newcommand{\HRule}{\rule{\linewidth}{0.5mm}}

%Definindo um comando todoin que aceita quebra de linha e fórmulas
\newcommand\todoin[2][]{\todo[inline, caption={2do}, #1]{
\begin{minipage}{\textwidth-4pt}#2\end{minipage}}}

\newcommand\todogeg[2][]{\todo[inline, caption={#2}, color=yellow!100, #1]{
\begin{minipage}{\textwidth-4pt}#2\end{minipage}}}

\newcommand\todovwcm[2][]{\todo[inline, caption={#2}, color=red!100, #1]{
\begin{minipage}{\textwidth-4pt}#2\end{minipage}}}

\begin{document}

\begin{titlepage}
\begin{center}

% logo
%\includegraphics[width=0.35\textwidth]{images/ufrpe_logo_preto.png}~

\begin{figure}[ht]
		\includegraphics[height=3cm]{images/usp_75.png}
		\hspace{3.5cm}
    	\includegraphics[height=2.5cm]{images/each.png}
	\end{figure}   


\textsc{\large ACH2118} \\
\small Introdução ao Processamento de Língua Natural\\[2cm]

% identificação do relatório
\HRule \\[0.4cm]
{\large \bfseries Relatório de Implementação do Classificador de Clareza para Respostas na Plataforma eSIC \\[0.4cm]}
\HRule 
\\[2cm]

% identificação do aluno
\large\textbf{Aluno}\\
Felipe Mateos Castro de Souza - 11796909\\[1cm]
Nilton Tadashi Enta - 12730911\\[1cm]

% identificação do orientador
\large\textbf{Docente}\\
Ivandré Paraboni\\
Escola de Artes, Ciências e Humanidades - EACH\\[1cm]



\vfill

% Bottom of the page
{\large \today}

\end{center}
\end{titlepage}

\newpage
\tableofcontents
\newpage
\section{Introdução}
\label{sec:introducao}

Este relatório apresenta os resultados da implementação de um classificador de nível de clareza para respostas publicadas na plataforma eSIC. O objetivo central do projeto foi desenvolver um modelo capaz de categorizar as respostas em três níveis de clareza, identificados como 'c1', 'c234' e 'c5'. O trabalho foi conduzido em duas partes distintas: a primeira envolveu o desenvolvimento do classificador e a apresentação dos resultados médios de acurácia utilizando validação cruzada de 10 partições sobre o conjunto de dados de treinamento. A segunda parte consistiu na geração de previsões para o conjunto de teste.

Ao longo deste processo, realizamos uma análise exploratória dos dados, aplicamos técnicas de pré-processamento, vetorização de texto e implementamos modelos de aprendizado de máquina clássicos e neurais. O relatório detalha cada etapa do desenvolvimento, desde a importação e análise dos dados até a avaliação final dos modelos no conjunto de teste.

Os algoritmos testados incluíram Naive Bayes, Support Vector Machine, Random Forest e LSTM, sendo que a otimização dos modelos foi realizada através do grid search para SVM e Random Forest. As métricas de avaliação, como acurácia média na validação cruzada e acurácia no conjunto de teste, foram registradas para cada algoritmo, proporcionando uma visão abrangente do desempenho de cada modelo.

Os resultados obtidos neste relatório são essenciais para a escolha do modelo mais adequado para a tarefa de classificação de respostas na plataforma eSIC, contribuindo para a eficácia na análise de clareza dessas respostas.

\section{Análise Exploratória}
\label{sec:analysis}

Realizamos uma análise preliminar do conjunto de dados utilizando as bibliotecas pandas e numpy.

Verificamos as informações gerais da base como presença de valores nulos, tipo de cada feature e quantidade de linhas. O dataframe possui 6 mil linhas e nenhuma vazia, todas do tipo 'object'. 

Em seguida agrupamos os dados por classe e partindo da função 'describe' verificamos a presença de dados duplicados. Cada classe possui um conjunto de 2 mil textos sendo únicos:

c1 - 1.839

c234 - 1.929

c5 - 1911

E analisamos também as palavras que se repetiam em cada classe. Mapeamos as dez maiores classes e em comum ambas tinham como as três palavras mais presente em seus textos: 'informação', 'por' e 'para'.
\section{Pré-processamento}
\label{sec:pre-processing}

\subsection{Tratamento dos dados}

No âmbito do tratamento de dados textuais, a análise precisa e eficiente requer uma série de técnicas avançadas. Abaixo descrevemos o processo que  utilizamos, abordando desde a tokenização até a codificação de rótulos.

a. Tokenização:
A primeira etapa envolve a tokenização, um processo fundamental que divide o texto em unidades semânticas chamadas tokens. Cada palavra ou expressão é isolada, facilitando análises subsequentes e fornecendo uma visão granular do conteúdo textual. Optamos por utilizar a biblioteca NLTK para esta tarefa.

b. Lemmatização e remoção de pontuações:
A lematização entra em cena para simplificar a análise ao reduzir palavras flexionadas às suas formas base. Essa técnica garante que diferentes formas de uma palavra sejam tratadas como uma única entidade, facilitando a compreensão e reduzindo a complexidade do conjunto de dados. Já para esta tarefa, optamos por usar a biblioteca SpaCy, onde fizemos uso do modelo `pt\_core\_news\_sm', vale ressaltar que juntamente desse processo, fizemos a remoção de pontuações, usando o tagger da biblioteca SpaCy.

c. POS Tagging (Etiquetagem de Partes da Fala):
A etiquetagem de partes da fala adiciona camadas de informações valiosas, atribuindo etiquetas a cada palavra de acordo com sua função gramatical. Essa abordagem aprimora a compreensão contextual, permitindo uma análise mais profunda das relações sintáticas e semânticas entre as palavras.

d. 2-Grams e 3-Grams:
A implementação de bigramas (2-grams) e trigramas (3-grams) expande a análise para além das palavras isoladas, considerando sequências de duas e três palavras consecutivas. Isso captura relações mais complexas e contextuais no texto, fornecendo insights mais ricos sobre a estrutura e o significado. Para essa tarefa optamos por utilizar a mesmo biblioteca que no tokenizer, isto é, a NLTK.

e. Label Encoding (Codificação de Rótulos):
A etapa final do processo consiste na codificação de rótulos, transformando informações textuais em representações numéricas. Essa técnica é crucial para a utilização eficiente de algoritmos de aprendizado de máquina, permitindo a aplicação de modelos preditivos a dados previamente tratados.

Para sermos mais eficientes, optamos por armazenar o tratamentos feitos em arquivos do tipo CSV, sendo o mais relevante deles o ``esic2023\_cleaned.csv".

\subsection{Separação do Conjunto de Dados}

Com a finalidade de aumentar nossa confiança nos resultados das métricas que seriam obtidas após o treinamento dos modelos, optamos por dividir o conjunto de dados em duas partes, um conjunto de treino, correspondente a 70\% dos dados, e um conjunto de teste, correspondente a 30\% dos dados.

Dessa forma, podemos realizar uma média de uma validação cruzada no co conjunto de treino e depois comparar seus resultados com a métrica de acurácia no conjunto de teste, para aferir se houve \emph{overfitting}.
 
\subsection{Técnicas de Vetorização de Texto}

Uma abordagem eficaz para essa tarefa é a utilização do método TF-IDF (Term Frequency-Inverse Document Frequency), e para implementar essa técnica de maneira robusta e eficiente, a biblioteca Scikit-Learn oferece ferramentas poderosas.

O Scikit-Learn fornece uma implementação flexível e fácil de usar do TF-IDF, permitindo que os usuários ajustem parâmetros conforme necessário e integrem facilmente essa etapa em seus pipelines de PLN e aprendizado de máquina. Desse modo optamos por utilizá-la em nosso projeto, com o hiperparâmetro `max\_features=5000' na coluna `lemma' dos conjuntos de treino e teste.

Optamos por armazenar as matrizes resultantes em arquivos NPZ e NPY, para os atributos e rótulos respesctivamente, nos poupando, então, de ter que gerá-los novamente tod vez que reiniciássemos o kernel do Jupyter.

\subsection{Técnicas de Word Embeddings}

Uma abordagem inovadora para essa tarefa é o uso do modelo BERTimbau, uma variação do BERT (Bidirectional Encoder Representations from Transformers), presente na renomada biblioteca transformers. Optamos por utilizá-lo na seguinte configuração: return\_tensors="pt", truncation=True, padding=True, max\_length=512, add\_special\_tokens = True. Pois foi a que conferiu maior resultado dentre as abordagens usando embeddings,
\section{Treinamento de modelos}
\label{sec:model_training}

\subsection{Teste de Modelos de classificadores}

Sob o data frame disponibilizado foram aplicados diversos classificadores, não neurais e neurais. Dentre os não neurais foi testado a Naive Bayes, Regressão Logística, Random Forest e SVM. Para os modelos neurais foi escolhido a LSTM e o BERT, ambos com uma rede neural como classificador.

As redes neurais demandam um tempo maior de processamento e apresentaram uma acurácia similar ou próxima das não neurais, porém com uma variação maior entre resultados. Sendo assim, escolhemos seguir com os algoritmos não neurais.


\subsection{Modelagem do Algoritmo}

Realizamos a otimização utilizando grid search do sklearn nos modelos de SVM e RF, escolhidos por ter a acurária maior dentre os modelos não neurais. Após algums testes e seleção de parâmetros o modelo que apresentou o melhor desempenho foi o SVM.


\section{Avaliação dos Modelos}
\label{sec:model_eval}

Após termos realizado a otimização de busca de hiperparâmetros usando o grid search da biblioteca sklearn nos algoritmos mais promissores, chegamos à conclusão que o algoritmo de SVM (Support Vector Machine) se saíria melhor que os demais (Naive Bayes, Random Forest e LSTM), principalmente usando a vetorização produzida pelo TFIDF.

Desse modo, nosso \emph{ensemble} final ficou com as seguintes características:

Vetorização: TFIDF dos textos lemmatizados e sem pontuação, com `max\_features'=5000
Algoritmo: SVM com `C'=2.0 e o `kernel'=rbf

\subsection{Conjunto de treinamento}

Dentro do conjunto de treinamento, que corresponde a 70\% dos do conjunto de dados, fizemos uma validação cruzada de 10 \emph{folds}. Após a obtençaõ da média dos valores dessa métrica, constatamos que o modelo obteve uma acurácia média de 56.55\%.

\subsection{Conjunto de teste}

Já no conjunto de teste, que corresponde a 30\% dos dados, realizamos um teste de acurácia para aferir se o modelo permanecia com resultados consistentes. Obtivemos o resultado de 58.17\% acurácia. Essa diferença aparenta ser pequena demais, e portanto, nos diz que o modelo não apresentou `overfitting'.
\section{Reprodução dos resultados}
\label{sec:results}

Basta seguir as instruções presentes no arquivo main.ipynb

\end{document}