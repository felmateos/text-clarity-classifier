\newpage
\section{Introdução}
\label{sec:introducao}

Este relatório apresenta os resultados da implementação de um classificador de nível de clareza para respostas publicadas na plataforma eSIC. O objetivo central do projeto foi desenvolver um modelo capaz de categorizar as respostas em três níveis de clareza, identificados como 'c1', 'c234' e 'c5'. O trabalho foi conduzido em duas partes distintas: a primeira envolveu o desenvolvimento do classificador e a apresentação dos resultados médios de acurácia utilizando validação cruzada de 10 partições sobre o conjunto de dados de treinamento. A segunda parte consistiu na geração de previsões para o conjunto de teste.

Ao longo deste processo, realizamos uma análise exploratória dos dados, aplicamos técnicas de pré-processamento, vetorização de texto e implementamos modelos de aprendizado de máquina clássicos e neurais. O relatório detalha cada etapa do desenvolvimento, desde a importação e análise dos dados até a avaliação final dos modelos no conjunto de teste.

Os algoritmos testados incluíram Naive Bayes, Support Vector Machine, Random Forest e LSTM, sendo que a otimização dos modelos foi realizada através do grid search para SVM e Random Forest. As métricas de avaliação, como acurácia média na validação cruzada e acurácia no conjunto de teste, foram registradas para cada algoritmo, proporcionando uma visão abrangente do desempenho de cada modelo.

Os resultados obtidos neste relatório são essenciais para a escolha do modelo mais adequado para a tarefa de classificação de respostas na plataforma eSIC, contribuindo para a eficácia na análise de clareza dessas respostas.
